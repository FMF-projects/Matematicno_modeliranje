\documentclass[A4paper, 11pt]{article}
\usepackage[slovene]{babel}
\usepackage[utf8]{inputenc}
\usepackage{array}
\usepackage{amsfonts}
\usepackage{pifont}
\usepackage{theorem}
\usepackage{authblk}
\usepackage{url}

\usepackage{graphicx}

\title{Problem simetrične diskretne verižnice z liho mnogo členki}
\author{Klementina Pirc}
\affil{Fakulteta za matematiko in fiziko \\ Oddelek za matematiko}
\date{julij 2020}

\newtheorem{definicija}{Definicija}
\theorembodyfont{\mdseries}
\newtheorem{zgled}{Zgled}

\newcommand{\cmark}{\ding{51}}
\newcommand{\xmark}{\ding{55}}

\begin{document}

\begin{titlepage} 

\maketitle
\thispagestyle{empty}
	
\end{titlepage}


% OPIS PROBLEMA

\section{Opis problema}

Imamo diskretno verižnico, to je verižnico sestavljeno iz palic, katerih dolžino in maso poznamo. Konca verižnice pritrdimo v točki $T1$ in $T2$. Zanima nas oblika verižnice, ki pri tem nastane, torej želimo izračunati koordinate stičišč palic. Vemo, da na verižnico deluje sila gravitacije, zato bo njena oblika takšna, da bo potencialna energija verižnice najmanjša možna. 

% vstavi sliko neke diskretne veriznice

V nadaljevanju bom predstavila postopek rešitve za poseben primer diskretne verižnice in sicer za simetrično diskretno verižnico z liho mnogo členki. Simetričnost pomeni, da sta točki $T1$ in $T2$ na enaki višini, ter da so dolžine in mase palic simetrične glede na sredinsko palico. Oglejmo si sedaj še matematično formulacijo problema.

Diskretna verižnica je sestavljena iz $2p+1$ palic, kjer je $p \in \mathbb{N}$. Poznamo dolžine in mase palic, torej $L_i$ in $M_i$ za $i=1 \ldots 2p+1$. Želimo izračunati koordinate krajišč palic $(x_i,y_i)$ za $i=1 \ldots 2p$, $(x_0,y_0)$ in $(x_{2p+1},y_{2p+1})$ pa že poznamo, saj sta to obesišči verižnice, torej $T1$ in $T2$. Zaradi simetričnosti velja $y_0 = y_{2p+1}$ ter $L_{2p+2-i} = L_i$ in $M_{2p+2-i} = M_i$ za $i=1 \ldots p$.

% slika simetricne z lihim stevilom clenov


% REŠEVANJE

\section{Reševanje problema}

Prvi del postopka je enak kot pri rešitvi za splošno diskretno verižnico, le da že upoštevamo liho število palic in pišemo $2p+1$ namesto $n+1$. 
Minimizirati želimo potencialno energijo verižnice oziroma sistema homogenih palic. Potencialno energijo posamezne palice opišemo z enačbo
\[ W_i = \frac{1}{2} \cdot M_i \cdot g \cdot (y_{i-1} + y_i) \quad za \quad i=1 \ldots 2p+1 \]
zato je potencialna energija celotne verižnice enaka
\[ W = \sum_{1}^{2p+1} W_i = \sum_{1}^{2p+1} \frac{1}{2} \cdot M_i \cdot g \cdot (y_{i-1} + y_i) \]
$g$ je gravitacijska konstanta in ne vpliva na minimum potencialne energije, zato jo lahko izpustimo in obravnavamo le funkcijo
\[ F(x,y) = \sum_{1}^{2p+1} \frac{1}{2} \cdot M_i \cdot (y_{i-1} + y_i) \]
Za iskane točke $(x_i,y_i)$ velja Pitagorov izrek $(x_i - x_{i-1})^2 + (y_i - y_{i-1})^2 = L_i ^2$, torej iščemo vezani ekstrem. Uporabimo Lagrangeovo metodo in problem prevedemo na reševanje (nevezanega) ekstrema funkcije
\[ G(x,y,\lambda) = \sum_1^{2p+1} \frac{1}{2} M_i (y_{i-1} - y_i) + \lambda_i ((x_i - x_{i-1})^2 + (y_i - y_{i-1})^2 - L_i ^2) \]

% REZULTATI

\section{Rezultati} 



% LITERATURA

\begin{thebibliography}{9}
% 1
\bibitem{w-quatum}
	Wikipedia, 2018. Quantum key distribution,  \\
	Dostopno na:
	\textit{\url{https://en.wikipedia.org/wiki/Quantum_key_distribution}}
	[24.3.2018]
\end{thebibliography}

\end{document}


